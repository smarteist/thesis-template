
\chapter{مقدمه}

نخستین فصل یک پایان‌نامه به معرفی مسئله، بیان اهمیت موضوع، ادبیات موضوع،
اهداف پژوهش و معرفی ساختار پایان‌نامه می‌پردازد.
در این فصل نمونه‌ی مختصری از مقدمه آورده شده است.

\section{تعریف مسئله}

نگارش یک پایان‌نامه‌ علاوه بر بخش‌های پژوهش و آماده‌سازی محتوا،
مستلزم رعایت نکات دقیق فنی و نگارشی است
که در تهیه‌ی یک پایان‌نامه‌ی موفق بسیار کلیدی و مؤثر است.
از آن جایی که بسیاری از نکات فنی مانند قالب کلی صفحات، شکل و اندازه‌ی قلم،
صفحات عنوان و غیره در تهیه‌ی پایان‌نامه‌ها یکسان است،
می‌توان با ارائه‌ی یک قالب حروف‌چینی استاندارد
نگارش پایان‌نامه‌ها را تا حد بسیار زیادی بهبود بخشید.

\section{اهمیت موضوع}

وجود قالب استاندارد برای نگارش پایان‌نامه از جهات مختلف حائز اهمیت است، از جمله:

\begin{itemize}
\item
ایجاد یک‌نواختی در قالب کلی صفحات و شکل و اندازه‌ی قلم‌ها
\item
تسهیل نگارش پایان‌نامه با در اختیار گذاشتن یک قالب اولیه
\item
تولید خودکار صفحات دارای بخش‌های تکراری نظیر صفحات ابتدایی و انتهایی پایان‌نامه
\item
پیش‌گیری از برخی خطاهای مرسوم در نگارش پایان‌نامه
\end{itemize}

\section{ادبیات موضوع}

اکثر دانشگاه‌های معتبر قالب استانداردی برای تهیه‌ی پایان‌نامه در اختیار دانشجویان خود قرار می‌دهند.
این قالب‌ها عموما مبتنی بر نرم‌افزارهای متداول حروف‌چینی نظیر لاتک و مایکروسافت ورد هستند.

 لاتک\پانویس{\LaTeX} یک نرم‌افزار متن‌باز قوی برای حروف‌چینی متون علمی است.\cite
 {knuth1984texbook, lamport1985LaTeX}
در این نوشتار از نرم‌افزار حروف‌چینی زی‌تک\پانویس{\XeTeX}
 و افزونه‌ی زی‌پرشین\پانویس{\XePersian}
 استفاده شده است.


\section{اهداف پژوهش}

کتابخانه‌ی مرکزی دانشگاه صنعتی شریف دستورالعمل جامعی را در خصوص
نحوه‌ی تهیه‌ی پایان‌نامه‌ی کارشناسی ارشد و رساله‌ی دکتری ارائه کرده است.
در این نوشتار سعی شده است قالب استانداردی برای تهیه‌ی پایان‌نامه‌ها مبتنی بر نرم‌افزار لاتک و
بر اساس دستورالعمل مذکور ارائه شده و
نحوه‌ی استفاده از قالب به طور مختصر توضیح داده شود.
این قالب  می‌تواند برای تهیه‌ی پایان‌نامه‌های کارشناسی و کارشناسی ارشد
و همچنین رساله‌ها‌ی دکتری مورد استفاده قرار گیرد.

\section{ساختار پایان‌نامه}

این پایان‌نامه در پنج فصل به شرح زیر ارائه می‌شود.
%فصل دوم به بیان مفاهیم اولیه  می‌پردازد.
نکات اولیه‌ی نگارشی و نحوه‌ی نگارش پایان‌نامه در محیط لاتک در  فصل دوم به اختصار اشاره شده است.
فصل سوم به مطالعه و بررسی کارهای پیشین مرتبط با موضوع این پایان‌نامه می‌پردازد.
در فصل چهارم، نتایج جدیدی که در این پایان‌نامه به‌دست آمده است، ارائه می‌شود.
فصل پنجم به جمع‌بندی کارهای انجام شده در این پژوهش و ارائه‌ی پیشنهادهایی برای انجام کارهای آتی خواهد پرداخت.